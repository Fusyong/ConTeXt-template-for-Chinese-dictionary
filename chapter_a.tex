\starttext
\startcolumns

\chapter{A}

\section{a}

\subsection{ā(ㄚ)}
\subsection{bā(ㄚ)}

%%%%%%%%%%%%%%%%%%test



%%%%%%%%%%%%%%%%%%s

\zi
{吖}
{ā}
{见下。}
{}

\ci
{吖嗪}
{āqín}
{名有机化合物的一类,呈环状结构,含有一个或几个氮原子,如吡啶、哒嗪、嘧啶等。[英 azine]}
{}

\zi
{阿}
{ā}
{〈方〉前缀。①用在排行、小名或姓的前面,有亲昵的意味:~大|~宝|~唐。②用在某些亲属称谓的前面:~婆|~爸|~哥。}
{\see{阿-•a};\see{阿-ē}。}

\ci
{阿鼻地狱}
{ābí dìyù}
{佛教指最深层的地狱,是犯了重罪的人死后灵魂永远受苦的地方。[阿鼻,梵 avīci]}
{}

\ci
{阿昌族}
{Āchānɡzú}
{名我国少数民族之一,分布在云南。}
{}

\ci
{阿斗}
{Ā Dǒu}
{〈名〉三国蜀汉后主刘禅的小名。阿斗为人庸碌,后多用来指懦弱无能的人。}
{}

\ci
{阿尔茨海默病}
{ā’ěrcíhǎimòbìnɡ}
{名老年性痴呆的一种,多发生于中年或老年的早期,因德国医生阿尔茨海默(Alois Alzheimer)最先描述而得名。症状是短期记忆丧失,认识能力退化,逐渐变得呆傻,以至生活完全不能自理。}
{}

\ci
{阿尔法粒子}
{ā’ěrfǎ lìzǐ}
{某些放射性物质衰变时放射出来的氦原子核,由两个中子和两个质子构成,质量为氢原子的4倍,带正电荷,穿透力不强。通常写作α粒子。[阿尔法,希腊字母α的音译]}
{}

\ci
{阿尔法射线}
{ā’ěrfǎ shèxiàn}
{放射性物质放射出来的阿尔法粒子流。通常写作α射线。}
{}

\ci
{阿飞}
{āfēi}
{名指身着奇装异服、举动轻狂、行为不端的青少年。}
{}


\ci
{阿伏伽德罗常量}
{Āfújiādéluó chánɡliànɡ}
{指1摩任何物质所含的粒子(分子、原子、离子等)数,约等于6.022×10²³。因纪念意大利化学家阿伏伽德罗(Amedeo Avogadro)而得名。旧称阿伏伽德罗常数。}
{}

\ci
{阿公}
{āɡōnɡ}
{方名①丈夫的父亲。②祖父。③尊称老年男子。}
{}

\ci
{阿訇}
{āhōnɡ}
{名我国伊斯兰教称主持清真寺教务和讲授经典的人。[波斯 ākhūnd]}
{}

\ci
{阿拉伯人}
{Ālābórén}
{名亚洲西南部和非洲北部的主要居民。原住阿拉伯半岛,多信伊斯兰教。[阿拉伯,阿拉伯语 Arab]}
{}

\ci
{阿拉伯数字}
{Ālābó shùzì}
{国际通用的数字,就是0、1、2、3、4、5、6、7、8、9。最初由印度人发明、使用,因后经阿拉伯人传入欧洲,所以叫阿拉伯数字。}
{}

\ci
{阿兰若}
{ālánrě}
{名原指树林、寂静处,后多指佛教寺庙。[梵 āraṇya]}
{}

\ci
{阿罗汉}
{āluóhàn}
{名见860页〖罗汉〗。[梵 arhat]}
{}

\ci
{阿猫阿狗}
{āmāo āɡǒu}
{方泛指某类人或随便什么人(含轻蔑意)。}
{}

\ci
{阿门}
{āmén}
{叹犹太教、基督教祈祷时常用的结束语,“但愿如此”的意思。[希伯来 āmen]}
{}

\ci
{阿片}
{āpiàn}
{名从尚未成熟的罂粟果里取出的乳状液体,干燥后变成淡黄色或棕色固体,味苦。医药上用作止泻药和镇痛药。用后容易成瘾,是一种毒品。用作毒品时,叫鸦片。}
{}

\ci
{阿婆}
{āpó}
{方名①丈夫的母亲。②祖母。③尊称老年妇女。}
{}

\zi
{阿}
{QĀ Q}
{名鲁迅小说《阿Q正传》的主人公,是“精神胜利者”的典型,受了屈辱,不敢正视,反而用自我安慰的办法,说自己是“胜利者”。}
{}

\ci
{阿是穴}
{āshìxué}
{名中医在针灸上把本无固定位置,而以明显的酸、麻、胀、痛等感觉确定的穴位,叫作阿是穴。}
{}

\ci
{阿嚏}
{ātì}
{拟声形容打喷嚏的声音。}
{}

\ci
{阿姨}
{āyí}
{名①方母亲的姐妹。②称呼跟母亲辈分相同、年纪差不多的无亲属关系的妇女:王~|售票员~。③对保育员或保姆的称呼。}
{}

\zi
{呵}
{ā}
{同“啊”(ā)。}
{\see{呵-á};\see{呵-ǎ};\see{呵-à};\see{呵-•a};\see{呵-hē};\see{呵-kē}。}

\zi
{啊}
{ā}
{叹表示惊异或赞叹:~,出虹了!|~,今年的庄稼长得真好哇!}
{\see{啊-á};\see{啊-ǎ};\see{啊-à};\see{啊-•a}。}

\zi
{锕錒}
{ā}
{名金属元素,符号Ac(actinium)。银白色,有放射性。锕-227用作航天器的热源。}
{}

\zi
{腌}
{ā}
{见下。}
{\see{腌-yano}}

\ci
{腌臜}
{ā•za}
{方①形脏;不干净;房子里太~了,快打扫打扫吧。②形(心里)别扭;不痛快:晚到一步,事没办成,~透了。③动糟践;使难堪:算了,别~人了。}
{}

\zi
{呵}
{á}
{同“啊”(á)。}
{\see{呵-ā};\see{呵-ǎ};\see{呵-à};\see{呵-•a};\see{呵-hē};\see{呵-kē}。}

\zi
{啊}
{á}
{叹表示追问:~?你明天到底去不去呀?|~?你说什么?}
{\see{啊-ā};\see{啊-ǎ};\see{啊-à};\see{啊-•a}。}

\zi
{嗄}
{á}
{同“啊”(á)。}
{\see{嗄-shà}。}

\zi
{呵}
{ǎ}
{同“啊”(ǎ)。}
{\see{呵-ā};\see{呵-á};\see{呵-à};\see{呵-•a};\see{呵-hē};\see{呵-kē}。}

\zi
{啊}
{ǎ}
{叹表示惊疑:~?怎么会有这种事?}
{\see{啊-ā};\see{啊-á};\see{啊-à};\see{啊-•a}。}

\zi
{呵}
{à}
{同“啊”(à)。}
{\see{啊-ā};\see{啊-á};\see{啊-ǎ};\see{啊-•a};\see{啊-hē};\see{啊-kē}。}

\zi
{啊}
{à}
{叹①表示应诺(音较短):~,好吧。②表示明白过来(音较长):~,原来是你,怪不得看着面熟哇!③表示赞叹或惊异(音较长):~,伟大的祖国!|~,真没想到他会取得这么好的成绩!}
{\see{啊-ā};\see{啊-á};\see{啊-ǎ};\see{啊-•a}。}

\zi
{阿}
{•a}
{同“啊”(•a)。}
{\see{ā};\see{ē}。}

\zi
{呵}
{•a}
{同“啊”(•a)。}
{\see{呵-ā};\see{呵-á};\see{呵-ǎ};\see{呵-à};\see{呵-hē};\see{呵-kē}。}

\zi
{啊}
{•a}
{助①用在感叹句末,表示增强语气:多好的天儿~!|他的行为多么高尚~!②用在陈述句末,使句子带上一层感情色彩:这话说得是~!|我也没说你全错了~!③用在祈使句末,使句子带有敦促或提醒意味:慢慢儿说,说清楚点儿~|你可别告诉小邓~!④用在疑问句末,使疑问语气舒缓些:他什么时候来~?|你吃不吃~?⑤用在句中稍作停顿,让人注意下面的话:这些年~,咱们的日子越过越好啦。⑥用在列举的事项之后:书~,报~,杂志~,摆满了一书架。⑦用在重复的动词后面,表示过程长:乡亲们盼~,盼~,终于盼到了这一天。注意(适用于以上义项)“啊”用在句末或句中,常受到前一字尾音的影响而发生不同的变音,书面上常按变音写成“呀、哇、哪”不同的字:
\starttable[|l|l|]
\HL
\NC 前字的韵母或韵尾   \VL  “啊”的发音和写法 \SR
\HL
\NC a,e,i,o,ü    \VL a→ia 呀 \AR
\NC u,ao,ou   \VL a→ua 哇 \AR
\NC -n   \VL a→na 哪 \AR
\NC -ng   \VL a→nga \LR
\stoptable}
{\see{啊-ā};\see{啊-á};\see{啊-ǎ};\see{啊-à}。}

\zi
{哎}
{āi}
{叹①表示惊讶或不满意:~!真是想不到的事|~!你怎么能这么说呢!②表示提醒:~,我倒有个办法,你们大家看行不行?}
{}

\ci
{哎呀}
{āiyā}
{叹①表示惊讶:~!这瓜长得这么大呀!②表示埋怨、不耐烦、惋惜、为难等:~,你怎么来得这么晚呢!|~,你就少说两句吧!|~,时间都白白浪费了|~,这事不好办哪!}
{}

\ci
{哎哟}
{āiyō}
{叹表示惊讶、痛苦、惋惜等:~!都十二点了!|~!我肚子好疼!|~,咱们怎么没有想到他呀!}
{}

\zi
{哀}
{āi}
{①悲伤;悲痛:悲~|~鸣|~叫|节~。②悼念:默~。③怜悯:~怜|~矜|~其不幸。④(Āi)名姓。}
{}

\ci
{哀兵必胜}
{āibīnɡ-bìshènɡ}
{《老子》六十九章:“故抗兵相若,则哀者胜矣。”对抗的两军力量相当,悲愤的一方获得胜利。指受压抑而奋起反抗的军队,必然能打胜仗。}
{}


















\stopcolumns
\stoptext
